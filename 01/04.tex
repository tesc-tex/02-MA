\subsection{%
  Интегрирование рациональных функций. Интегрирование простейших дробей 1,2,3.%
} \label{int-basic-frac}

\begin{itemize}
\item Интегрирование простейших дробей \(I\)-ого типа

\begin{align*}
  \int \frac{A}{x - a}\dd x
  = A \int \frac{\dd (x - a)}{x - a}
  = \ln \abs{x - a} + C
\end{align*}

\item Интегрирование простейших дробей \(II\)-ого типа

\begin{align*}
  \int \frac{A}{(x - a)^{k}}\dd x
  = A \int (x - a)^{-k} \dd (x - a)
  = \frac{A}{1 - k} \cdot (x - a)^{1 - k}
  =\frac{A}{1 - k} \cdot \frac{1}{(x - a)^{k - 1}}
\end{align*}

\item Интегрирование простейших дробей \(III\)-его типа

\begin{align}\label{eq:int-basic-frac-3}\tag{1}
  \int \frac{Mx + N}{x^2 + px + q} \dd x
\end{align}

Попытаемся внести числитель под дифференциал:

\begin{align*}
  \dd (x^2 + px + q) = (2x + p) \dd x \\
  (Mx + N)
  = \frac{M}{2} \left( 2x + \frac{2N}{M} \right)
  = \frac{M}{2} \left( 2x + p + \frac{2N}{M} - p \right)
  = \frac{M}{2} (2x + p) + \under{\left( N - \frac{Mp}{2} \right)}{h}
\end{align*}

Подставим это в \eqref{eq:int-basic-frac-3}:

\begin{align*}
  \int \frac{\frac{M}{2} (2x + p) + h}{x^2 + px + q} \dd x =
  \frac{M}{2} \cdot \int \frac{2x + p}{x^2 + px + q} \dd x + \int \frac{h}{x^2 + px + q} \dd x
\end{align*}

Далее вычислим каждый из интегралов по-отдельности:

\begin{align*}
  \frac{M}{2} \cdot \int \frac{2x + p}{x^2 + px + q} \dd x =
  \frac{M}{2} \cdot \int \frac{\dd (x^2 + px + q)}{x^2 + px + q} =
  \frac{M}{2} \cdot \ln \abs{x^2 + px + q} + C
  \\
  \int \frac{h}{x^2 + px + q} \dd x =
  h \cdot \int \frac{1}{(x + \sfrac{p}{2})^2 +
    \under{q - (\sfrac{p}{2})^2}{g^2}
  } \dd x =
  \frac{h}{g} \cdot \arctg \left(\frac{x + \sfrac{p}{2}}{g}\right) + C
\end{align*}

Подставим полученные выражения в исходный интеграл \eqref{eq:int-basic-frac-3}:

\begin{align*}
  \int \frac{Mx + N}{x^2 + px + q} \dd x =  
  \frac{M}{2} \cdot \ln \abs{x^2 + px + q} +
  \frac{h}{g} \cdot \arctg \left(\frac{x + \sfrac{p}{2}}{g}\right) + C \\
  h = \left(N - \frac{Mp}{2} \right), g^2 = q - \left(\frac{p}{2}\right)^2
\end{align*}

\item Интегрирование простейших дробей \(IV\)-его типа

\begin{example}
  \begin{align*}\label{eq:int-basic-frac-4-1}\tag{1}
    \int \frac{\dd x}{(x^2 + 1)^2} =
    \int \frac{x^2 + 1 - x^2}{(x^2 + 1)^2} \dd x =
    \int \frac{\dd x}{x^2 + 1} - \int \frac{x^2}{(x^2 + 1)^2} \dd x
  \end{align*}
  
  Первый из полученных интегралов мы уже вычислить, это простейшая дробь
  \(III\)-ого типа. Таким образом этот интеграл будет равен \(\arctg x + C\).
  Далее работаем со вторым интегралом:
  
  \begin{align*}\label{eq:int-basic-frac-4-2}\tag{2}
    \int \frac{x^2}{(x^2 + 1)^2} \dd x =
    \frac{1}{2} \cdot \int \frac{x \dd (x^2 + 1)}{(x^2 + 1)^2} =
    \left[
      \frac{\dd t}{t^2} = -\dd \left(\frac{1}{t}\right)
    \right] =
    -\frac{1}{2} \cdot \int x \cdot \dd \left( \frac{1}{x^2 + 1} \right) 
  \end{align*}
  
  Полученный интеграл возьмем по частям:
  
  \begin{align*}\label{eq:int-basic-frac-4-3}\tag{3}
    \int \under{x}{u} \dd \under{\left(\frac{1}{x^2 + 1}\right)}{v} =
    \frac{x}{x^2 + 1} - \int \frac{\dd x}{x^2 + 1} =
    \frac{x}{x^2 + 1} - \arctg x + C 
  \end{align*}
  
  Подставим \eqref{eq:int-basic-frac-4-3} в \eqref{eq:int-basic-frac-4-2}, а
  полученное выражение в исходный интеграл \eqref{eq:int-basic-frac-4-1}:
  
  \begin{align*}
    \int \frac{\dd x}{(x^2 + 1)^2} =
    \int \frac{\dd x}{x^2 + 1} - \int \frac{x^2}{(x^2 + 1)^2} \dd x = \\
    \arctg x + \frac{1}{2} \left(\frac{x}{x^2 + 1} - \arctg x\right) + C = \\
    \frac{1}{2} \arctg x + \frac{x}{2 (x^2 + 1)} + C
  \end{align*}
  
  \begin{remark}
    В случаях с более высокой степенью каждая подобная итерация будет приводить
    к уменьшению степени знаменателя на единицу. Обычно подобные интегралы
    раскладываются с помощью подведения под дифференциал и линейности, после
    чего используется следующая рекуррентная формула:

    \begin{align*}
      I_{n}
      = \int \frac{\dd x}{(x^2 + a^2)^n}
      = \frac{x}{2 a^2 (n - 1) (x^2 + a^2)^{n - 1}}
        + \frac{2n - 3}{2 a^2 (n - 1)} \cdot I_{n - 1}
    \end{align*}
  \end{remark}
\end{example}

\end{itemize}
