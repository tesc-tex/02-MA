\subsection{%
  Геометрический смысл определенного интеграла. Оценка определенного интеграла.
  Теорема о среднем.%
}

Геометрический смысл определенного интеграла следует из его построения:
определенный интеграл по модулю равен площади криволинейной трапеции.

\begin{lemma}\label{int-est}
  Пусть \(f \in C_{[a;b]}\) и определен \(\int_{a}^{b} f(x) \dd x\).
  \(m\), \(M\)~--- наименьшее и наибольшее значения функции \(f(x)\) на
  отрезке \([a;b]\). Тогда

  \begin{align*}
    (b - a) m \le \int_{a}^{b} f(x) \dd x \le (b - a) M
  \end{align*}
\end{lemma}
\begin{proof}
  \begin{align*}
    \forall x \in [a;b] \colon m \le f(x) \le M
    \implies \forall \xi_{i} \in [a;b] \colon m \le f(\xi_{i}) \le M
    \\
    m \Delta x_{i} \le f(\xi_{i}) \Delta x_{i} \le M \Delta x_{i}
    \\
      m \sum_{i = 1}^{n} \Delta x_{i}
    \le 
      \sum_{i = 1}^{n} f(\xi_{i}) \Delta x_{i}
    \le 
      M \sum_{i = 1}^{n} \Delta x_{i}
    \\
      m \lim_{\substack{n \to \infty \\ \tau \to 0}}
      \sum_{i = 1}^{n} \Delta x_{i}
    \le 
      \lim_{\substack{n \to \infty \\ \tau \to 0}}
      \sum_{i = 1}^{n} f(\xi_{i}) \Delta x_{i}
    \le 
      M \lim_{\substack{n \to \infty \\ \tau \to 0}}
      \sum_{i = 1}^{n} \Delta x_{i}
    \\
    (b - a) m \le \int_{a}^{b} f(x) \dd x \le (b - a) M
  \end{align*}
\end{proof}

\begin{theorem}\label{L-mid-int}
  Теорема Лагранжа о среднем (в интегральной форме)

  Пусть \(f \in C_{[a;b]}\) и определен \(\int_{a}^{b} f(x) \dd x\). Тогда

  \begin{align*}
    \exists \xi \in (a; b) \colon \int_{a}^{b} f(x) \dd x = f(\xi) (b - a)
  \end{align*}
\end{theorem}
\begin{proof}
  Воспользуемся леммой \ref{int-est}:

  \begin{align*}
    (b - a) m \le \int_{a}^{b} f(x) \dd x \le (b - a) M
    \implies m \le \frac{1}{b - a} \cdot \int_{a}^{b} f(x) \dd x \le M
  \end{align*}

  По т. Больцано-Коши функция \(f(x)\) принимает все значения от минимального
  \(m\) до максимального \(M\). Значит \(\exists \xi \in (a; b)\), что

  \begin{align*}
    f(\xi) = \frac{1}{b - a} \cdot \int_{a}^{b} f(x) \dd x
    \implies \int_{a}^{b} f(x) \dd x = f(\xi) (b - a)
  \end{align*}
\end{proof}

\begin{remark}
  Геометрический смысл теоремы Лагранжа заключается в том, что на промежутке
  \((a; b)\) всегда найдется такая точка \(\xi\), что площадь криволинейной
  трапеции будет в точности равна площади прямоугольника со сторонами
  \((b - a)\) и \(f(\xi)\).
\end{remark}

\begin{lemma}
  Если \(f(x), g(x) \in C_{[a; b]}\), определены
  \(\int_{a}^{b} f(x) \dd x\), \(\int_{a}^{b} g(x) \dd x\)
  и при этом
  \(\forall x \in [a; b] \colon f(x) \ge g(x)\), то

  \begin{align*}
    \int_{a}^{b} f(x) \dd x \ge \int_{a}^{b} g(x) \dd x
  \end{align*}
\end{lemma}
\begin{proof}
  Рассмотрим \(h(x) = f(x) - g(x)\). Она будет неотрицательная на отрезке
  \([a; b]\), значит \(\int_{a}^{b} h(x) \dd x \ge 0\). Далее пользуемся
  аддитивноcтью и получаем искомое неравенство.
\end{proof}

\begin{lemma}\label{dint-abs-prop}
  Пусть \(f \in C_{[a;b]}\) и определен \(\int_{a}^{b} f(x) \dd x\). Тогда

  \begin{align*}
    \abs{\int_{a}^{b} f(x) \dd x} \le \int_{a}^{b} \abs{f(x)} \dd x
  \end{align*}
\end{lemma}
\begin{proof}
  Т.к. определенный интеграл это предел интегральных сумм, то можно
  воспользоваться предельных переходом, а затем свойством о том, что модуль
  суммы не превосходит сумму модулей.
\end{proof}

\begin{remark}
  Выкалывание из отрезка \([a; b]\) конечного числа точек не меняет значение
  интеграла.
\end{remark}
