\subsection{%
  Сходимость интегралов 1-го и 2-го рода от степенных функций.%
}

\begin{definition}
  Интегралы, про сходимость которых известно, называются \textit{эталонными}.
  Обычно они используются в признаках сравнения.
\end{definition}

Исследуем на сходимость интеграл
\(\displaystyle \int_{1}^{+\infty} \frac{\dd x}{x^{\alpha}}\).
Рассмотрим три случая:

\begin{align*}
  \alpha = 1 &
  \qquad
  \int_{1}^{+\infty} \frac{\dd x}{x}
  = \ln \abs{x} \bigg\vert_{1}^{+\infty}
  & \implies \int_{1}^{+\infty} \frac{\dd x}{x} \notconverge
  \\
  \alpha > 1 &
  \qquad
  \int_{1}^{+\infty} \frac{\dd x}{x^{\alpha}}
  = \left(\frac{x^{1 - \alpha}}{1 - \alpha}\right) \bigg\vert_{1}^{+\infty}
  = \frac{1}{1 - \alpha} \lim_{x \to \infty} x^{1 - \alpha}
    - \frac{1}{1 - \alpha}
    & \implies \int_{1}^{+\infty} \frac{\dd x}{x^{\alpha}} \converge
  \\
  \alpha < 1 &
  \qquad
  \int_{1}^{+\infty} \frac{\dd x}{x^{\alpha}}
  = \left(\frac{x^{1 - \alpha}}{1 - \alpha}\right) \bigg\vert_{1}^{+\infty}
  = \frac{1}{1 - \alpha} \lim_{x \to \infty} x^{1 - \alpha}
    - \frac{1}{1 - \alpha}
  & \implies \int_{1}^{+\infty} \frac{\dd x}{x^{\alpha}} \notconverge
\end{align*}

Исследуем на сходимость интеграл
\(\displaystyle \int_{a}^{b} \frac{\dd x}{(x - a)^{\alpha}}\).
Также рассмотрим три случая:

\begin{align*}
  \alpha = 1 &
  \qquad
  \int_{a}^{b} \frac{\dd x}{(x - a)}
  = \ln \abs{x - a} \bigg\vert_{a}^{b}
  & \implies \int_{a}^{b} \frac{\dd x}{x - a} \notconverge
  \\
  \alpha > 1 &
  \qquad
  \int_{a}^{b} \frac{\dd x}{(x - a)^{\alpha}}
  = \frac{(x - a)^{1 - \alpha}}{1 - \alpha} \bigg\vert_{a}^{b}
  = \frac{(b - a)^{1 - \alpha}}{1 - \alpha} -
    \lim_{x \to a+} \frac{(x - a)^{1 - \alpha}}{1 - \alpha}
  & \implies \int_{a}^{b} \frac{\dd x}{(x - a)^{\alpha}} \notconverge
  \\
  \alpha < 1 &
  \qquad
  \int_{a}^{b} \frac{\dd x}{(x - a)^{\alpha}}
  = \frac{(x - a)^{1 - \alpha}}{1 - \alpha} \bigg\vert_{a}^{b}
  = \frac{(b - a)^{1 - \alpha}}{1 - \alpha} -
    \lim_{x \to a+} \frac{(x - a)^{1 - \alpha}}{1 - \alpha}
  & \implies \int_{a}^{b} \frac{\dd x}{(x - a)^{\alpha}} \converge
\end{align*}

Аналогично можно исследовать сходимость интеграла
\(\displaystyle \int_{a}^{b} \frac{\dd x}{(b - x)^{\alpha}}\).
Таким образом получаем:

\begin{align*}
  \int_{1}^{+\infty} \frac{\dd x}{x^{\alpha}} \converge \; \alpha > 1 
  \qquad
  \int_{a}^{b} \frac{\dd x}{(x - a)^{\alpha}} \converge \; \alpha < 1
  \qquad
  \int_{a}^{b} \frac{\dd x}{(b - x)^{\alpha}} \converge \; \alpha < 1
  \\
  \int_{a}^{+\infty} \frac{\dd x}{x^{\alpha}} \notconverge \; \alpha \le 1
  \qquad
  \int_{a}^{b} \frac{\dd x}{(x - a)^{\alpha}} \notconverge \; \alpha \ge 1
  \qquad
  \int_{a}^{b} \frac{\dd x}{(b - x)^{\alpha}} \notconverge \; \alpha \ge 1
\end{align*}

\begin{remark}
  Как правило для проверки на сходимость интегралов разного вида используют
  разные эталонные интегралы:

  \begin{align*}
    \int_{a}^{+\infty} f(x) \dd x
      \longrightarrow \int_{1}^{+\infty} \frac{\dd x}{x^{\alpha}}
    \\
    \int_{a}^{b} f(x) \dd x
      \longrightarrow \int_{a}^{b} \frac{\dd x}{(x - a)^{\alpha}}
  \end{align*}
\end{remark}
