\subsection{%
  Признаки сходимости несобственных интегралов: первый признак сравнения (в
  неравенствах).%
}

\begin{theorem}\label{cnv-cmp}
  Пусть \(f(x), g(x) \colon [a, +\infty] \to \RR\) и на этом отрезке выполняется
  неравенство \(f(x) \ge g(x) \ge 0\). Тогда:

  \begin{align}
    \int_{a}^{+\infty} f(x) \dd x \converge
    \implies \int_{a}^{+\infty} g(x) \dd x \converge
    \label{eq:cnv-cmp-a}\tag{a}
    \\
    \int_{a}^{+\infty} g(x) \dd x \notconverge
    \implies \int_{a}^{+\infty} f(x) \dd x \notconverge
    \label{eq:cnv-cmp-b}\tag{b}
  \end{align}
\end{theorem}
\begin{proof}
  \eqref{eq:cnv-cmp-a} Сначала докажем первое утверждение.
  Т.к. \(f(x) \ge 0\), то
  \(I = \int_{a}^{b} f(x) \dd x \ge 0 \in \RR\),
  при этом т.к. этот интеграл сходится, то \(I \in \RR\).
  Далее рассмотрим второй интеграл, по определению имеем:

  \begin{align*}
    \int_{a}^{+\infty} g(x) \dd x
    = \lim_{\beta \to +\infty}
      \under{\int_{a}^{\beta} g(x) \dd x}{h(\beta)}
  \end{align*}

  Заметим, т.к. \(g(x) \ge 0\), то функция \(h(\beta)\) монотонно возрастает при
  \(\beta \to +\infty\). При этом значение этой функции ограничено сверху
  числом \(I \in \RR\). Значит по свойствам пределов данный предел конечен, из
  чего следует, что интеграл \(\int_{a}^{+\infty} g(x) \dd x\) сходится.

  \eqref{eq:cnv-cmp-b} Доказательство второго утверждения вытекает из первого.
  От противного: пусть \(\int_{a}^{+\infty} f(x)\) сходится. Тогда по
  пункту \ref{eq:cnv-cmp-a} интеграл \(\int_{a}^{+\infty} g(x)\) тоже должен
  сходится. Противоречие.
\end{proof}
