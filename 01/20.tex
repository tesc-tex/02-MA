\subsection{%
  Несобственные интегралы 2-го рода (от неограниченной функции). Определение,
  вычисление и свойства.%
}

\begin{definition}
  Пусть \(f(x) \in C_{[a; b]}\) и \(b\) это точка бесконечного разрыва
  (\(\lim_{x \to b} f(x) = \infty\)), тогда интеграл 

  \begin{align*}
    \int_{a}^{b} f(x) \dd x
    \bydef
    \lim_{\beta \to b-} \int_{a}^{\beta} f(x) \dd x
  \end{align*}

  называется несобственным интегралом 2-ого рода.
\end{definition}

\begin{remark}
  Существуют также другие формы несобственных интегралов 2-ого рода:

  \begin{align*}
    \int_{a}^{b} f(x) \dd x
    \bydef
    \lim_{\alpha \to a+} \int_{\alpha}^{b} f(x) \dd x
    \\
    \int_{a}^{b} f(x) \dd x
    \bydef
    \int_{a}^{c} f(x) \dd x + \int_{c}^{b} f(x) \dd x
  \end{align*}

  В первом случае точкой бесконечного разрыва является точка \(a\), а во 
  втором~--- \(c \in (a; b)\).
\end{remark}

Несобственные интегралы второго рода обладают теми же свойствами (линейность,
аддитивность, сравнение) и вычисляются так же, как и несобственные интегралы
1-ого рода:

\begin{align*}
  \int_{a}^{b} f(x) \dd x = \lim_{\beta \to b} F(x) \bigg\vert_{a}^{\beta}
\end{align*}
