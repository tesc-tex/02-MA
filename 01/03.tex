\subsection{%
  Интегрирование рациональных функций (общая схема). Разложение дроби на
  простейшие.%
}

Выделим \(4\) типа простейших дробей:

\begin{align*}
  (I): \quad \frac{A}{x - a} \qquad
  (II): \quad \frac{A}{(x - a)^k} \qquad
  (III): \quad \frac{Mx + N}{x^2 + px + q} \qquad
  (IV): \quad \frac{Mx + N}{(x^2 + px + q)^{k}} \qquad
\end{align*}

где \((x^2 + px + q)\) неразложимый на множители многочлен, а \(A, M, N\)~---
неопределенные коэффициенты.

\underline{Метод неопределенных коэффициентов}:

Пусть дана дробь \(\dfrac{Q_{m}(x)}{P_{n}(n)}\), в которой \(Q_{m}(x)\) и
\(P_{n}(x)\) это многочлены с вещественными коэффициентами. Требуется
разложить её на простейшие.

\begin{enumerate}
  \item Если \(m \ge n\), то необходимо выделить целую часть. Далее будем
  считать, что \(m < n\).
  
  \item Раскладываем знаменатель на множители, т.е. приводим его к виду
  
  \begin{align*}
    P_{n} = a_{0}
      (x - x_{1})^{b_{1}} \dotsc (x - x_{t})^{b_{t}}
      (x^2 + p_{1} x + q_{1})^{c_{1}} \dots (x^2 + p_{r} x + q_{r})^{c_{r}}
  \end{align*}

  \item Для каждой скобки в знаменателе записываем некоторую дробь по
  следующему правилу:
  
  \begin{align*}
    (x - x_{i}) & \to & \frac{A}{x - x_{i}}
    \\
    (x - x_{i})^{k} & \to & \frac{A}{x - x_{i}}
      + \dotsc + \frac{A_k}{(x - x_{i})^{k}}
    \\
    (x^2 + p_{i} x + q_{i}) & \to & \frac{Ax + B}{x^2 + p_{i} x + q_{i}}
    \\
    (x^2 + p_{i} x + q_{i})^{k} & \to & \frac{Ax + B}{x^2 + p_{i} x + q_{i}}
      + \dotsc + \frac{A_{k}x + B_{k}}{(x^2 + p_{i} x + q_{i})^{k}}
  \end{align*}

  У каждой скобки будет свой набор констант.

  \item Получаем уравнение относительно коэффициентов \(A, B \dots\), которые
  находятся в числителе полученных дробей.

  \item Приводим полученную дробь к общему знаменателю и приравниваем её к
  исходной дроби.

  \item Т.к. знаменатели полученных дробей равны, то должны быть равны и
  числители. Пользуемся тем, что два полинома равны когда равны все коэффициенты
  перед одинаковыми степенями. Получаем систему уравнений (по количеству
  коэффициентов).

  \item Решаем систему, находим коэффициенты. Подставляем их в разложение
  исходной дроби на сумму простейших дробей.
\end{enumerate}

Теперь интегрирование рациональных дробей свелось к тому, чтобы разложить их
на простейшие, а потом, пользуясь линейностью интеграла (\ref{ad-prop-3}),
проинтегрировать каждую из дробей по-отдельности.
Подробнее об интегрировании простейших дробей написано в вопросе
\ref{int-basic-frac}

