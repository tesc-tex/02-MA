\subsection{%
  Определение и вычисление тройного интеграла.%
}

Пусть в \(\RR^3\) есть область, в которой определена скалярная величина и
'плотность' её распределения \(\rho(x, y, z)\). Тогда содержание этой величины
в данной области будет равно:

\begin{align*}
  \iiint_{T} \rho(x, y, z) \under{\dd x \dd y \dd z}{\dd V}
\end{align*}

\begin{remark}
  В определении выше рассматривается область, правильная в направлении \(Oz\).
\end{remark}

\begin{remark}
  Свойства тройного интеграла, а также способ его вычисления полностью
  аналогичен двойному интегралу:
  \begin{enumerate}
    \item Определяем границы для одной из переменных.

    \item Выражаем границы для второй переменной через первую, а для третьей~---
    через первые две.

    \item Сводим все к повторным интегралам.
  \end{enumerate}
\end{remark}

Формула для вычисления тройного интеграла будет выглядеть следующим образом:

\begin{align*}
  \iiint_{T} f(x, y, z) \dd x \dd y \dd z
  = \int_{x_{1}}^{x_{2}} \dd x
    \int_{y_{1}(x)}^{y_{2}(x)} \dd y
    \int_{z_{1}(x, y)}^{z_{2}(x, y)} f(x, y, z) \dd z
\end{align*}
