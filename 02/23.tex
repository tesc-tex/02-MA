\subsection{%
  Векторная запись теорем теории поля и их механический смысл.%
}

\begin{theorem}
  О потенциале

  \begin{align*}
    \int_{L} P \dd x + Q \dd y + R \dd z \text{ НЗП}
    \iff
    \oint_{K} = 0
    \iff
    \begin{Bmatrix}
      R'_{y} = Q'_{z} \\
      P'_{z} = R'_{x} \\
      Q'_{x} = P_{y}
    \end{Bmatrix}
    \iff
    \exists u(x, y, z) \colon \grad u = \vec{F}
  \end{align*}
\end{theorem}
\begin{proof}
  Как было показано ранее (\ref{potential-field}) последнее равенство равносильно
  тому, что \(\rot \vec{F} = 0\). Таким образом

  \begin{align*}
    \rot \vec{F} = 0 = \oint_{K} = \circulation
  \end{align*}
\end{proof}

\begin{theorem}
  Теорема Стокса.

  \begin{align*}
    \iint_{S} \rot \vec{F} \cdot \vec{n_{0}} \cdot \dd \sigma
    = \int_{L} \vec{F} \dd l = \circulation
  \end{align*}
\end{theorem}
\begin{proof}
  Теорема Стокса в координатной форме уже доказана (\ref{ST}). Запишем её:

  \begin{align*}
    \iint_{S}
      \under{\left(
        \frac{\partial R}{\partial y} 
        - \frac{\partial Q}{\partial z}
      \right)}{\rot \vec{F}_{x}} \cos \alpha \dd \sigma
      + \under{\left(
        \frac{\partial P}{\partial z} 
        - \frac{\partial R}{\partial z}
      \right)}{\rot \vec{F}_{y}} \cos \beta \dd \sigma
      + \under{\left(
        \frac{\partial Q}{\partial x} 
        - \frac{\partial P}{\partial y}
      \right)}{\rot \vec{F}_{z}} \cos \gamma \dd \sigma
    = \oint_{L^{+}} P \dd x + Q \dd y + R \dd z
  \end{align*}

  Заметим, что в скобках перед косинусами находятся соответствующие проекции
  ротора на координатные оси. Учитывая то, что
  \(\vec{n_{0}} = (\cos \alpha, \cos \beta, \cos \gamma)\), получаем:
  
  \begin{align*}
    \iint_{S} \rot \vec{F} \cdot \vec{n_{0}} \dd \sigma
    = \int_{L^{+}} \vec{F} \dd \vec{l}
  \end{align*}
\end{proof}

\begin{theorem}
  Теорема Гаусса-Остроградского.

  \begin{align*}
    \iiint_{T} \div \vec{F} \dd V
    =
    \oiint_{S_{T}} \vec{F} \cdot \vec{n_{0}} \dd \sigma
  \end{align*}
\end{theorem}
\begin{proof}
  Теорема Гаусса-Остроградского в координатной форме уже доказана (\ref{GO}).
  Запишем её:

  \begin{align*}
    \iiint_{T} \left(
      \frac{\partial P}{\partial x} +
      \frac{\partial Q}{\partial y} +
      \frac{\partial R}{\partial z}
    \right) \dd x \dd y \dd z
    =
    \oiint_{S_{T}} \left(
      P \cos \alpha +
      Q \cos \beta +
      R \cos \gamma
    \right) \dd \sigma
  \end{align*}

  Заметим, что под тройным интегралом в левой части выражения находится
  дивергенция поля \(\vec{F}\). Тогда

  \begin{align*}
    \iiint_{T} \div \vec{F} \dd v
    =
    \oiint_{S_{T}} \vec{F} \cdot \vec{n_{0}} \dd \sigma
  \end{align*}
\end{proof}

\begin{remark}
  Эти три теоремы устанавливают связь между содержанием величин внутри области и
  их расходом на границе области. Таким образом они все являются вариациями
  формулы Ньютона-Лейбница.
\end{remark}
