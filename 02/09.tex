\subsection{%
  Теорема (формула) Грина.%
}

\begin{twocolumns}
  \begin{theorem}\label{Green}
    Теорема Грина

    Пусть \(D\) правильная \(\uparrow Ox, \uparrow Oy\), \(\Gamma_{D} = K\)
    
    Даны функции \(P(x, y), Q(x, y) \colon K, D \to \RR\)

    Определен \(\displaystyle \oint_{K^{+}} P \dd x + Q \dd y\)
    
    Тогда

    \begin{align*}
      \boxed{
        \oint_{K^{+}} P \dd x + Q \dd y
        = \iint_{D}^{} \left(
          \frac{\partial Q}{\partial x} -
          \frac{\partial P}{\partial y}
        \right) \dd x \dd y
      }
    \end{align*}
  \end{theorem}
  \vfill\null
  \columnbreak

  \begin{tikzpicture}
  \draw[very thin, gray!30, step = 1cm] (-0.5, -0.5) grid (7.5, 5.5);

  \draw[thick] [->] (0, 0) -- (7, 0) node[right, below] {\(x\)};
  \draw[thick] [->] (0, 0) -- (0, 5) node[above, left] {\(y\)};

  \begin{scope}[
    decoration = {
      markings,
      mark = at position 0.85 with {\arrow[line width = 1pt]{>}}
    }
  ]
    \draw[postaction = { decorate }] (3.5, 2.5) ellipse (2.5cm and 1.5cm);
  \end{scope}

  \draw node[below] at (1, 0) {\(a\)};
  \draw node[below] at (6, 0) {\(b\)};
  \draw[dashed] (1, 0) -- (1, 2.5);
  \draw[dashed] (6, 0) -- (6, 2.5);

  \draw node[left] at (0, 1) {\(\alpha\)};
  \draw node[left] at (0, 4) {\(\beta\)};
  \draw[dashed] (0, 1) -- (4, 1);
  \draw[dashed] (0, 4) -- (4, 4);

  \draw node[left] at (1, 2.5) {\(M\)};
  \draw node[above] at (3.5, 4) {\(L\)};
  \draw node[right] at (6, 2.5) {\(N\)};
  \draw node[below] at (3.5, 1) {\(A\)};

  \draw[draw = none] (6, 2.5) arc (0 : 180 : 2.5cm and 1.5cm)
    node[pos = 0.4, below, sloped] {\(y_{2}(x)\)};
  \draw[draw = none] (1, 2.5) arc (180 : 360 : 2.5cm and 1.5cm)
    node[pos = 0.4, above, sloped] {\(y_{1}(x)\)};

  \draw[draw = none] (3.5, 1) arc (-90 : 90 : 2.5cm and 1.5cm)
    node[pos = 0.3, above, sloped] {\(x_{2}(y)\)};
  \draw[draw = none] (3.5, 4) arc (90 : 270 : 2.5cm and 1.5cm)
    node[pos = 0.3, below, sloped] {\(x_{1}(y)\)};
  
  \draw node at (3.5, 2.5) {\(D\)};
  \draw node at (5, 1) {\(K^{+}\)};

\end{tikzpicture}

\end{twocolumns}

\begin{proof}
  Рассмотрим двойной интеграл
  \(\displaystyle \iint_{D} \frac{\partial P}{\partial y} \dd x \dd y\)
  и сведем его к повторному:

  \begin{align*}
    \iint_{D} \frac{\partial P}{\partial y} \dd x \dd y
    = \int_{a}^{b} \dd x \int_{y_{1}(x)}^{y_{2}(x)}
      \frac{\partial P}{\partial y} \dd y
    = \int_{a}^{b} \left(
      P(x, y) \bigg\vert_{y_{1}(x)}^{y_{2}(x)}
    \right) \dd x
    = \int_{a}^{b} P(x, y_{2}(x)) \dd x
      - \int_{a}^{b} P(x, y_{1}(x)) \dd x 
  \end{align*}

  Используя формулу вычисления криволинейного интеграла 2-ого рода
  (\ref{curve-int-2-calc}) в обратную сторону получаем:

  \begin{align*}
    \int_{a}^{b} P(x, y_{2}(x)) \dd x - \int_{a}^{b} P(x, y_{1}(x)) \dd x = \\
    \int_{MLN} P(x, y) \dd x - \int_{MAN} P(x, y) \dd x = \\
    -\int_{NLM} P(x, y) \dd x - \int_{MAN} P(x, y) \dd x = \\
    -\oint_{K^{+}} P(x, y) \dd x 
  \end{align*}

  Аналогично можно показать, что \(\displaystyle
    \iint_{D} \frac{\partial Q}{\partial x} \dd x \dd y
    = \oint_{K^{+}} Q(x, y) \dd y 
  \). Объединяя эти равенства получаем, что

  \begin{align*}
    \oint_{K^{+}} P(x, y) \dd x + Q(x, y) \dd y
    = \oint_{K^{+}} Q (x, y) \dd y - \left(-\oint_{K^{+}} P(x, y) \dd x \right)
    = \iint_{D} \left(
      \frac{\partial Q}{\partial x}
      - \frac{\partial P}{\partial y}
    \right) \dd x \dd y
  \end{align*}
\end{proof}

\begin{remark}
  Формула Грина работает в обе стороны. Вычисляется тот интеграл, который проще.
\end{remark}

\begin{corollary}
  С помощью формулы Грина можно получить формулу для площади фигуры через
  криволинейный интеграл 2-ого рода:

  \begin{align*}
    S = \iint_{D} \dd x \dd y = \begin{bmatrix}
      \begin{rcases}
        P = -\sfrac{y}{2} \\
        Q = \sfrac{x}{2}
      \end{rcases}
      \implies Q'_{x} - P'_{y} = 1
    \end{bmatrix}
    = \oint_{K^{+}} -\frac{y}{2} \dd x + \frac{x}{2} \dd y
    = \frac{1}{2} \oint_{K^{+}} x \dd y - y \dd x
  \end{align*}
\end{corollary}
