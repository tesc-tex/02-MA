\subsection{%
  Следствие теоремы о независимости от пути (формула Ньютона-Лейбница).%
}

\begin{theorem}
  Формула Ньютона-Лейбница для интегралов, не зависящих от пути интегрирования

  Пусть выполнены условия теоремы \ref{path-ind-cr}, тогда

  \begin{align*}
    \int_{AB} P \dd x + Q \dd y = \Phi(B) - \Phi(A)
  \end{align*}
\end{theorem}
\begin{proof}
  Параметризуем дугу \(\breve{AB}\):

  \begin{align*}
    x = \phi(t) \implies \dd x = x'_{t} \dd t \\
    y = \psi(t) \implies \dd y = y'_{t} \dd t \\
    t \in [t_{1}; t_{2}] \iff A \to B
  \end{align*}

  Подставим это в исходный интеграл:

  \begin{align*}
    \int_{AB} P \dd x + Q \dd y
    = \int_{AB} \frac{\partial \Phi}{\partial x} \dd x
      + \frac{\partial \Phi}{\partial y} \dd y
    = \int_{t_{1}}^{t_{2}} \left(
      \frac{\partial \Phi}{\partial x} \cdot \frac{\dd x}{\dd t} 
      + \frac{\partial \Phi}{\partial y} \cdot \frac{\dd y}{\dd t} 
    \right) \dd t  
  \end{align*}

  Заметим, что то, что стоит в скобках, это полная производная \(\Phi\) по
  \(t\). Тогда имеем:

  \begin{align*}
    \int_{t_{1}}^{t_{2}} \frac{\dd \Phi(x(t), y(t))}{\dd t} \dd t
    = \Phi \Big( x(t), y(t) \Big) \bigg\vert_{t_{1}}^{t_{2}}
    = \Phi(B) - \Phi(A)
  \end{align*}
\end{proof}

