\subsection{%
  Связь между поверхностными интегралами 1-го и 2-го рода.%
}

Связь между поверхностными интегралами первого и второго рода была получена
при нахождении формулы для поверхностного интеграла 2-ого рода в проекциях
(\ref{surf-int-coords}):

\begin{align*}
  \under{
    \iint_{S} (\vec{F} \cdot \vec{n_{0}}) \dd \sigma
  }{II \text{ род}}
  = \under{
    \iint_{S} (P \cos \alpha + Q \cos \beta + R \cos \gamma) \dd \sigma
  }{I \text{ род}}
\end{align*}

Формулы для нахождения направляющих косинусов также же были получены в
предыдущем вопросе:

\begin{align*}\label{eq:surf-angles}\tag{ANG}
  \cos \alpha = \frac{\mp z'_{x}}{\sqrt{1 + (z'_{x})^2 + (z'_{y})^2}}
  \qquad
  \cos \beta= \frac{\mp z'_{y}}{\sqrt{1 + (z'_{x})^2 + (z'_{y})^2}}
  \qquad
  \cos \gamma = \frac{\pm 1}{\sqrt{1 + (z'_{x})^2 + (z'_{y})^2}}
\end{align*}

